\documentclass{article}
\usepackage{graphicx}
\usepackage{hyperref, url}
\usepackage{cite}
\usepackage{algorithm}
\usepackage{algpseudocode}
\usepackage[letterpaper]{geometry} 

\title{A Learning Augmented approach to Cardinality Estimation}
\author{Mărcuș Alexandru Marian}

\begin{document}
\maketitle
\section{Schita cuprins}
Abstract\\
Introduction\\
Related Work\\
Theoretical Foundation\\
Methodology\\
Implementation\\
Experiments\\
Conclusions\\


\section{Theoretical Foundation}




consistency, robustness, competitiveness

loss function: log loss/ cross entropy
$ L = (\log(E)-\log(N))^2 $

\section{Plan pentru partea aplicativa a lucrarii}
Implementarea standard a algoritmelor HLL si HLL++.\\
Generare de date cu diverse distributii ce ar putea fi regasite in date reale (ex. uniform, clustered etc).\\
Feature extraction din HLL sketch (ex. max, min, media, devia standard, \% din registre sunt goale, histograma a valorilor din registrii).\\
Antrenarea unui model mic (ex. regresie, un neural network cu putine layere etc).\\
Evaluarea modelului fata de standarde ca si acuratete, runtime si memorie.\\
Analiza rezultatelor si concluzii.\\


\section{Methodology}
\subsection{Overview}
We implement the standard HLL algorithm and augment it with a learned post-processing 
module. The neural component receives various features from the final register
array $M$ and uses them for deciding the weight of each register in the final result 
computation.

\subsection{Data}

\subsection{}

\begin{algorithm}[H]
\caption{Learned HyperLogLog}
\begin{algorithmic}[1]
\Require Let $h : \mathcal{D} \to \{0,1\}^{64}$ hash data from domain $\mathcal{D}$.  
Let $m = 2^p$ with $p \in [4..16]$.

\Statex
\textbf{Phase 0: Initialization.}
\State Define $\alpha_{16} = 0.673,\ \alpha_{32} = 0.697,\ \alpha_{64} = 0.709,$
\State \hspace{1.5em} $\alpha_m = 0.7213/(1 + 1.079/m)$ for $m \ge 128$.
\State Initialize $m$ registers $M[0]$ to $M[m-1]$ to 0.

\Statex
\textbf{Phase 1: Aggregation.}
\ForAll{$v \in S$}
    \State $x := h(v)$
    \State $\mathit{id} := (x_{63},\ldots,x_{64-p})_2$ \Comment{First $p$ bits of $x$}
    \State $w := (x_{63-p},\ldots,x_0)_2$
    \State $M[\mathit{id}] := \max(M[\mathit{id}], \rho(w))$
\EndFor

\Statex
\textbf{Phase 2: Result computation.}
\State \Return $E := \alpha_m m^2 \left( \sum_{j=0}^{m-1} 2^{-M[j]} \right)^{-1}$ \Comment{The “raw” estimate}

\end{algorithmic}
\end{algorithm}



\nocite{*}

\bibliographystyle{unsrt}
\bibliography{references}
\end{document}